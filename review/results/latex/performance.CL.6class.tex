\begin{table}[ht]
    \caption{Classical Learning accuracy of 6-class approaches }
    \label{tab:CL_performance_6class}
    \centering
    \begin{tabular}{m{.3\textwidth} l r r r r l}
        \toprule
        \textbf{Study} & \textbf{Best Method} & \textbf{Accuracy} & \textbf{F1} & \textbf{AUC} & \textbf{IR$^*$} & \textbf{Lang.} \\ 
        \midrule
        Włodzimierz et al.~\cite{Lewoniewski2016_lr18} & Random Forest & - & - & 0.90 & 1.00 & Multiple \\
        Vittoria et al.~\cite{Cozza2016_lr92} & Random Forest & - & - & 0.89 & 95.06 & English \\
        Schmidt and Zangerle~\cite{Schmidt2019_lr78} & Gradient Boosted Trees & 73.00\% & - & - & 1.10 & English \\
        Dang and Ignat~\cite{Dang2016_lr16} & Random Forest & 64.00\% & - & - & 1.77 & English \\
        Halfaker~\cite{Halfaker2017_lr22} & ORES (Gradient Boosting) & 62.90\% & - & - & 1.12 & English \\
        Halfaker and Geiger~\cite{Halfaker2020_lr1055} & Gradient Boosting & 62.90\% & - & - & 1.10 & Multiple \\
        Narun et al.~\cite{Raman2020_lr64} & MLR & 49.35\% & - & - & 1.00 & English \\
        \bottomrule
    \end{tabular}
    \\ \vspace{0.1cm}
    \footnotesize
    $^*$ Imbalance Ratio ($IR$) = \# samples in the majority class / \# samples in the minority class. 
\end{table}